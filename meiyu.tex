%%% REWRITE OF meiyu.tex FOR RESUBMISSION TO GEOPHYSICAL RESEARCH LETTERS, BEGUN MAY 29 2015

%%%%%%%%%%%%%%%%%%%%%%%%%%%%%%%%%%%%%%%%%%%%%%%%%%%%%%%%%%%%%%%%%%%%%%%%%%%%
% AGUtmpl.tex: this template file is for articles formatted with LaTeX2e,
% Modified March 2013
%
% This template includes commands and instructions
% given in the order necessary to produce a final output that will
% satisfy AGU requirements.
%
% PLEASE DO NOT USE YOUR OWN MACROS
% DO NOT USE \newcommand, \renewcommand, or \def.
%
% FOR FIGURES, DO NOT USE \psfrag or \subfigure.
%
%%%%%%%%%%%%%%%%%%%%%%%%%%%%%%%%%%%%%%%%%%%%%%%%%%%%%%%%%%%%%%%%%%%%%%%%%%%%
%
% All questions should be e-mailed to latex@agu.org.
%
%%%%%%%%%%%%%%%%%%%%%%%%%%%%%%%%%%%%%%%%%%%%%%%%%%%%%%%%%%%%%%%%%%%%%%%%%%%%
%
% Step 1: Set the \documentclass
%
% There are two options for article format: two column (default)
% and draft.
%
% PLEASE USE THE DRAFT OPTION TO SUBMIT YOUR PAPERS.
% The draft option produces double spaced output.
%
% Choose the journal abbreviation for the journal you are
% submitting to:

% jgrga JOURNAL OF GEOPHYSICAL RESEARCH
% gbc   GLOBAL BIOCHEMICAL CYCLES
% grl   GEOPHYSICAL RESEARCH LETTERS
% pal   PALEOCEANOGRAPHY
% ras   RADIO SCIENCE
% rog   REVIEWS OF GEOPHYSICS
% tec   TECTONICS
% wrr   WATER RESOURCES RESEARCH
% gc    GEOCHEMISTRY, GEOPHYSICS, GEOSYSTEMS
% sw    SPACE WEATHER
% ms    JAMES
% ef    EARTH'S FUTURE
%
%
%
% (If you are submitting to a journal other than jgrga,
% substitute the initials of the journal for "jgrga" below.)

\documentclass[draft,grl]{AGUTeX}
% To create numbered lines:

% If you don't already have lineno.sty, you can download it from
% http://www.ctan.org/tex-archive/macros/latex/contrib/ednotes/
% (or search the internet for lineno.sty ctan), available at TeX Archive Network (CTAN).
% Take care that you always use the latest version.

% To activate the commands, uncomment \usepackage{lineno}
% and \linenumbers*[1]command, below:

\usepackage{lineno}
\usepackage{amssymb}
\usepackage{amsmath}
\usepackage{textcomp}
\usepackage[super]{nth}
\usepackage{tabularx}
\usepackage{multirow}
\usepackage{bm}
\usepackage{color} 
\usepackage{hyperref}

\linenumbers*[1]

%  To add line numbers to lines with equations:
%  \begin{linenomath*}
%  \begin{equation}
%  \end{equation}
%  \end{linenomath*}
%%%%%%%%%%%%%%%%%%%%%%%%%%%%%%%%%%%%%%%%%%%%%%%%%%%%%%%%%%%%%%%%%%%%%%%%%
% Figures and Tables
%
%
% DO NOT USE \psfrag or \subfigure commands.
%
%  Figures and tables should be placed AT THE END OF THE ARTICLE,
%  after the references.
%
%  Uncomment the following command to include .eps files
%  (comment out this line for draft format):
\usepackage[final]{graphicx}
%
%  Uncomment the following command to allow illustrations to print
%   when using Draft:
%  \setkeys{Gin}{draft=false}
%
% Substitute one of the following for [dvips] above
% if you are using a different driver program and want to
% proof your illustrations on your machine:
%
% [xdvi], [dvipdf], [dvipsone], [dviwindo], [emtex], [dviwin],
% [pctexps],  [pctexwin],  [pctexhp],  [pctex32], [truetex], [tcidvi],
% [oztex], [textures]
%
% See how to enter figures and tables at the end of the article, after
% references.
%
%% ------------------------------------------------------------------------ %%
%
%  ENTER PREAMBLE
%
%% ------------------------------------------------------------------------ %%

% Author names in capital letters:
\authorrunninghead{DAY ET AL.}

% Shorter version of title entered in capital letters:
\titlerunninghead{STATISTICS OF THE SOUTH FLOOD-NORTH DROUGHT}

%Corresponding author mailing address and e-mail address:
\authoraddr{Corresponding author: Jesse Day, University of California Berkeley, Department of Earth and Planetary Science, College of Letters and Science; 307 McCone Hall, Berkeley, CA 94720, USA. (jessed@berkeley.edu)}

\begin{document}

%% ------------------------------------------------------------------------ %%
%
%  TITLE
%
%% ------------------------------------------------------------------------ %%


\title{Statistics of the South Flood-North Drought: A new catalog of rainbands in the East Asian monsoon}

%% ------------------------------------------------------------------------ %%
%
%  AUTHORS AND AFFILIATIONS
%
%% ------------------------------------------------------------------------ %%


%Use \author{\altaffilmark{}} and \altaffiltext{}

% \altaffilmark will produce footnote;
% matching \altaffiltext will appear at bottom of page.

\authors{Jesse A. Day\altaffilmark{1}, Inez Fung \altaffilmark{1}, and Weihan Liu\altaffilmark{2}}

\altaffiltext{1}{Department of Earth and Planetary Science, University of California Berkeley, Berkeley, California, USA.}
\altaffiltext{2}{College of Letters and Science, University of California Berkeley, Berkeley, California, USA.}


%% ------------------------------------------------------------------------ %%
%
%  ABSTRACT
%
%% ------------------------------------------------------------------------ %%

% >> Do NOT include any \begin...\end commands within
% >> the body of the abstract.

%Needs to be 150 words or less - currently at 147.
\begin{abstract}
A novel 57-year (1951-2007) daily catalog of frontal rainbands over China is compiled based on APHRODITE rain gauge data, resulting in an unprecedented climatology of Meiyu front progression in summer. Late \nth{20}-century changes in Chinese summer rainfall are investigated (the ``South Flood-North Drought''). Two robust changes in front behavior are observed during 1980-2007 relative to 1951-1979: 1) A significant decrease in the frequency of frontal rainbands during the Pre-Meiyu period (May), and 2) a southward shift in Post-Meiyu rainbands (mid-July to September). In contrast, the years 1994-2007 are marked by an increase in frontal intensity during Meiyu season relative to 1979-1993, but not frequency. By finding changes in the component of China rainfall associated with the large-scale atmosphere, our results begin to address the critical question of whether the South Flood-North Drought will persist under \nth{21}-century global warming.
\end{abstract}

%% Allowed length of manuscript is (# of words/500) + # of figures + # of tables.

%currently: 3489 words + 5 figures (max words: 3500)

\begin{article}

\section{Introduction}
 
 	Eastern China receives about 60\% of its rainfall from May to August via the East Asian summer monsoon. The period of peak rainfall lasting from early June to mid-July is called ``Meiyu season'' (lit. ``plum rains,'' referring to the spectacular growth of plum blossoms in central China with the onset of heavy rains). During this time, heavy rainfall occurs in zonal bands resulting from frontal synoptic conditions (the ``Meiyu front''). The rainfall climatology of Japan and Korea also features similar phenomena known as Baiu and Changma respectively. A growing volume of evidence suggests a shift in rainfall over China beginning in the late 1970s, featuring a ``South Flood-North Drought'' pattern shown in Figure \ref{changes_2d}a \citep{Hu1997,Gong2002,Nigam2013}. Due to the severe human impacts of the South Flood-North Drought on densely-populated eastern China, it is vital to understand whether this pattern will strengthen under global warming, or represents only a temporary deviation from the mean.
 
	The climatology of the East Asian monsoon bears little resemblance to other monsoon circulations \citep{Ding2005}. Whereas understanding of tropical monsoons has progressed greatly via theoretical studies \citep{Plumb1992,Prive2007,Bordoni2008}, the dynamics that favor the existence of frontal convection over East Asia in summer remain a point of debate, centering around the interplay of the tropospheric jet and Tibetan Plateau \citep{Molnar2010,Sampe2010,Chen2014}. Therefore, no simple conceptual template exists for interpreting a change such as the South Flood-North Drought. However, it is known that the migration of the Meiyu front entails a series of large-scale circulation changes \citep{Chen2004}, and furthermore that anomalies in Meiyu front latitude produce corresponding rainfall anomalies \citep{Kosaka2011}. Therefore, the South Flood-North Drought should be describable in terms of changes in the properties of Meiyu rainbands, such as a shift in latitude, a change in intensity or an earlier or delayed northward migration. In turn, such a characterization may provide insight into the dynamics responsible for the change.
	
	In pursuit of this aim, we have developed a 57-year database (1951-2007) of frontal rainbands in China based on the APHRODITE rain gauge product (described below). We develop a recursive convergent fitting algorithm of daily rainfall maps which finds rainbands and quantifies their attributes. Previous studies have investigated the statistics of the Meiyu front on decadal and even centennial timescales \citep{Chen2004,Ge2008,Xu2009}, but to our knowledge no previous author has compiled a multi-decadal daily catalog of events. We use this catalog to clarify the spatial and temporal attributes of the South Flood-North Drought, and present it as a tool for future East Asian monsoon research. We also expect that the South Flood-North Drought corresponds to larger-scale late-twentieth-century climatic changes, in particular the tropospheric jet, which plays an essential and complex role in East Asian summer climate \citep{Molnar2010}. We will further study changes in the jet in a subsequent companion paper.
		
\section{Data and Methods}

\subsection{APHRODITE}

	The APHRO\_MA\_V1101 product from APHRODITE (Asian Precipitation - Highly-Resolved Observational Data Integration Towards Evaluation of the Water Resources) includes 57 years (1951-2007) of daily rainfall (PRECIP product) on a .25\textdegree\ $\times$ .25\textdegree\ grid over 60-150\textdegree E and 15\textdegree S-55\textdegree N \citep{Yatagai2012}. Values are assimilated from weather station observations and therefore available over land only. We focus on the subregion inside of 100\textdegree E-123\textdegree E and 20\textdegree N-40\textdegree N, where Meiyu rainbands occur. Stations in this region are spaced at 100-200 km intervals (shown by RSTN product), such that rainbands are clearly resolved. APHRODITE's resolution cannot capture some features visible in TRMM satellite data \citep{Xu2009}, but its length allows for the study of decadal change.
	
\subsection{Rainband Detection Algorithm}

\subsubsection{Overview}

	For each day from 1 January 1951 to 31 December 2007 (20,819 days total), our recursive convergent image processing algorithm determines whether a rainband, defined as a continuous chain of rainfall maxima exceeding 10 mm day$^{-1}$ spanning at least 5\textdegree of longitude, exists inside the window of 105-123\textdegree E and 20-40\textdegree N. Properties of the rainband are calculated including latitude, intensity, tilt, length and width, as well as a ``quality score'' $Q$, defined as the fraction of daily rainfall occurring within the band. Fits with $Q<.6$ are discarded. We also test for the existence of two rainbands on a single day, an arrangement commonly found in August and September. In such a case, the first and second fitted rainbands are referred to as ``primary'' and ``secondary'' rainbands respectively. Our algorithm does not distinguish between the mechanisms that supply rainfall. Metrics of algorithm performance are documented in Supplementary Tables S1-S3. 
	
\subsubsection{Recursive Convergent Image Processing}

\begin{enumerate}
	\item Given a map of daily accumulated rainfall over the longitudes 105-123$^{\circ}$E and 20-40$^{\circ}$N at $.25^{\circ}$ by $.25^{\circ}$ resolution, the daily rainfall maximum at each longitude is found, and its intensity and latitude recorded. If there exists a $5^{\circ}$ continuous chain of maxima (20 points in a row) exceeding 10 mm day$^{-1}$, we proceed to step 2 and attempt a fit (Figure S1a). Otherwise, no fit is attempted for that day (Figure S1b).
	
	\item In order to approximate the position of the rainband with a best fit line, a weighted least-squares linear fit of the \textit{latitudes} of the maxima is performed in MATLAB using the intensity of each maximum as weight. To encourage convergence, the weight of outlying points is set to zero, where an outlier is defined as any maximum that is over $5^{\circ}$ from the centroid of the precipitation maxima $\left<lat_{max}\right>$, calculated by $\left<lat_{max}\right>=\frac{\sum_{long} lat_{max}*max}{\sum_{long} \max}$.
	
	\item A recursive algorithm converges on a best estimate of rainband position. In each iteration, we find a new set of maxima within \textit{k} degrees of the previous best fit line, and again perform a weighted linear fit of the maxima (Figure S2a). $k$ is progressively decreased with each iteration from $5^{\circ}$ to $2^{\circ}$ by $.25^{\circ}$ increments, and then from $2^{\circ}$ to $.25^{\circ}$ by $.25^{\circ}$ increments but repeating each width $k$ twice in a row (Figures S2b-c). The fit obtained in the final iteration is taken as our best estimate (Figure S2d).
	
	\item We define the ``quality score'' $Q$ as the fraction of total daily precipitation inside of  that falls within $2.5^{\circ}$ degrees of the best estimate line (Figure 4b). Other rainband properties are calculated as follows. Rainband latitude is defined as the latitude of our fitted band at the reference longitude of 115$^{\circ}$E. Intensity is defined as mean rainfall at all points along the band axis where daily rainfall exceeds 5 mm day$^{-1}$ (``rainband points''), length as the total number of rainband points (expressed in units of degrees of longitude), and width as the mean distance between half-maxima on either side of each rainband point (units of degrees of latitude).
	
	\item Given an estimate of primary rainband, we check for a secondary rainband. We start by removing all precipitation associated with the primary rainband from our daily rainfall map. To do this, all rainfall within 4$^{\circ}$ of our primary rainband is set to 0, as well as rainfall at any other adjacent points where rainfall exceeds 10 mm day$^{-1}$ (Figure S3a). We then reapply the continuous maximum criterion from step 1 (Figure S3b). If passed, steps 2-4 are repeated to find a best estimate for the position of the secondary rainband, and its attributes calculated.
	
	\item If a secondary rainband is found, two additional quality scores $Q_1$ and $Q_2$ are calculated. $Q_1$ is defined as the fraction of daily rainfall inside of 105-123$^{\circ}$E and 20-40$^{\circ}$N that is contained within $2.5^{\circ}$ degrees of the primary rainband \textit{after removing all rainfall associated with the secondary rainband} (equivalently, the $Q$ score after removing the secondary rainband). Likewise, $Q_2$ is the $Q$ score of the second rainband \textit{after removing all rainfall from the primary rainband} (Figure S4d).		
	
\end{enumerate} 

In rare cases with two rainbands of roughly equal strength but well-separated in latitude, the removal of outliers in step 2 prevents a fit entirely. We test for such cases by ensuring that the total sum of weights ${\sum_{long} max}$ exceeds 200 mm day$^{-1}$, equivalent to our condition in bullet point 1. When this condition is failed, which can only occur when too many of our maxima have been discarded, we return to step 1, but find maxima only over the latitude range 20N-$\left<lat_{max}\right>$ or $\left<lat_{max}\right>$-40N, depending on which half of our domain has a longer chain of maxima exceeding 10 mm day$^{-1}$, and apply the remaining steps of our algorithm as usual.

\subsubsection{Quality Control}

After running the algorithm for all 20,819 days from 1 January 1951 to 31 December 2007, we obtained 11,228 days with at least one rainband and 1,116 days with two rainbands. Subsequently, we apply a quality control (QC) algorithm to eliminate days with poor fit, based on the quality scores $Q$, $Q_1$ and $Q_2$ as well as the ``Taiwan fraction'' (TW), defined as the percentage of daily rainfall inside the window 105-$123^{\circ}$E and 20-$40^{\circ}$N that falls over the island of Taiwan (roughly 120-$122^{\circ}$E and 22-$26^{\circ}$N. We use these metrics to define the following two criteria for inclusion:

\begin{enumerate}

	\item If $TW > 20\%$, the day's fit is thrown out (238 cases total, 2.1\% of total fits). Such days are dominated by a local storm reaching Taiwan and do not exhibit a strong rainband (Figure 4a).  
	
	\item Subsequently, a day can be included if it meets either of the following criteria:
	
	\begin{enumerate} 
	
	\item if $Q>.6$, the day is included in our statistics (7,522 days, 67.0\% of total fits; Figure 4b). If $Q_2$ is also greater than .6, the day will be classified as a double rainband day (Type I double rainband; 232 cases, or 3.1\% of days where $Q>.6$ and $Q_2>.6$).
		
	\item If $Q<.6$, the day is discarded unless two fronts are detected and both $Q_1 \mathrm{\textbf{ and }} Q_2 > .6$ (where again $Q_1$ and $Q_2$ are \textit{conditional} quality scores as defined above). In such cases, the presence of multiple rainbands of similar intensity initially obscures the quality of the fit (Figure 4d). Such days are also classified as double rainband days (Type II double rainband; 466 cases). In the absence of such a double rainband configuration the day is thrown out (Figure 4c).
	
	\end{enumerate}
	
\end{enumerate}	

	The use of conditional quality scores $Q_1$ and $Q_2$ adds 466 double rainband days that would otherwise have been categorized as days with no rainband, constituting 6.2\% of total days included in our statistics. 33.2\% of double rainband days are Type I ($Q>.6$) and 66.8\% Type II ($Q<.6$). Double rainbands are more common during certain seasons such as the Post-Meiyu (see Figure 1c in main text). Tables S1-S3 contain more information on algorithm functionality.
	
\subsection{Bootstrapping and Significance of Changes}

	The standard deviation and significance of changes in rainband frequency are calculated analytically. However, rainband statistics do not follow a normal distribution. (INSERT CITATION ABOUT USE OF GAMMA DISTRIBUTION?). We use bootstrapping with replacement to calculate the standard deviation of their means (Figure \ref{jet_seasonal}a and Supplementary Tables S4-S8 respectively). To calculate the statistical significance of differences in mean between two samples, two methods are used: 1) Bootstrapping with replacement and 2) a permutation test (bootstrapping without replacement) \citep{Good2005}. Both produce similar results; $p$-values shown are from permutation testing. Figure 3a uses a moving blocks bootstrap with block length of 3 days (Supplementary Text S3). We focus on changes in front attributes between 1951-1979 and 1980-2007 (Supplementary Tables S5 and S6), and also repeat our methodology for 1979-1993 versus 1994-2007, since other authors have found a significant shift between these sets of years (CITE?).
	
\section{Rainband Statistics}	
	
\subsection{Climatology}	

%NEW PARAGRAPH DESCRIBING FRONT PERCENTAGE FIGURE
	Rainbands contribute a substantial fraction of yearly rainfall throughout our study region, from a maximum of 72\% over the Yangtze River Valley to a minimum of 12\% in the arid northern interior of China. Rainband frequency also decreases to  15\% on the eastern coast of Taiwan, where many local mechanisms deliver rainfall \citep{Chen2003}.

	The yearly progression of precipitation over eastern China is shown in Figure \ref{hov}a, longitudinally averaged over $100-123^\circ$E with a 5-day running mean, similar to Figure 7 in \citet{Ding2005}. China receives a substantial fraction of its yearly precipitation outside of summer, unlike other monsoonal regions \citep{Wang2002}. Figure \ref{hov}b shows a Hovm\"oller diagram of rainband frequency over all 57 years, including both primary and secondary rainbands. Some periods of heavy rainfall, in particular the August peak over southern China (over 10 mm day$^{-1}$ around 20\textdegree N), do not correspond to a surge in rainbands. Figure \ref{hov}c shows the probability of observing a rainband and mean rainband intensity, and Figure \ref{hov}d shows mean rainband tilt and length, as well as the conditional probability of observing a secondary rainband given the presence of a primary rainband. Frontal rainbands over China can appear in any month, with their probability of occurrence and intensity maximizing in late June (80\% probability of occurrence, mean intensity of 31 mm day$^{-1}$) and minimizing in January (10\% probability occurrence, mean intensity of 12 mm day$^{-1}$).
	
	Coordinated, abrupt changes occur in rainfall and frontal climatology. We define 5 periods of notable behavior as demarcated in Figure \ref{hov}: 1) The ``Spring Rains'' (days 60-120, March 1-April 30), as previously studied in \citet{Tian1998}; 2) The ``Pre-Meiyu'' (121-160, May 1-June 9), during which rainfall and front intensity increase; 3) Meiyu season (161-200, June 10-July 19) when a remarkable 7-degree northward shift in mean rainband latitude occurs over the course of several weeks, and rainband frequency and intensity peaks; 4) The Post-Meiyu (201-273, July 20-September 30), during which double rainbands are common; and 5) the ``Fall Rains'' (274-320, October 1-November 16), when rainband latitude returns south. The Pre-Meiyu, Meiyu and Post-Meiyu are equivalent to the three stages of Meiyu rainfall described in \citet{Ding2005}. Our results can also be compared with the event catalog of \citet{Xu2009}, which finds a similar date for the northward transition of the Meiyu front. The total number of rainband counts as well as the mean and standard deviation of rainband frequency, latitude and intensity during each time period are presented in Supplementary Table S4.  In addition, Figures \ref{climo}a-e show mean rainfall, jet frequency and rainband position during each stage, as well as their zonal average (sidebars). From the Pre-Meiyu to Post-Meiyu, each northward jump in peak rainband frequency corresponds to a similar shift in jet count density, with a southward offset of about 5 degrees. A yearly asymmetry can also be seen between more frequent and intense rainbands during the jet's northward passage (Pre-Meiyu and Meiyu) versus weaker rainfall during its southward return (Fall Rains), which merits further study.
		
\subsection{Changes in Rainband Attributes, 1980-2007 Versus 1951-1979}
	
	We calculate changes in rainfall and rainband frequency during 1980-2007 relative to 1951-1979, along with their statistical significance (Figure \ref{changes}). In addition, we evaluate the significance of changes in rainband attributes between these sets of years during each of the five rainfall stages (Supplementary Tables S5 and S6). Finally, Figures \ref{changes_2d}b and \ref{changes_2d}c show spatial changes in rainfall and jet count density during the Pre-Meiyu and Post-Meiyu, when changes are particularly large. During the Pre-Meiyu (days 121-160), the probability of observing a primary rainband has declined from $59.0\% \pm 2.0\%$ to $53.0\% \pm 2.1\%$ ($p=0.020$; Table S5). A corresponding decrease in Pre-Meiyu rainfall has occurred in central China (Figure \ref{changes_2d}b and 30\textdegree N in Figure \ref{changes}a). The change in Pre-Meiyu rainfall in the late \nth{20} century has previously been reported by \citet{Xin2006} and \citet{Wang2009}.
		
	In addition, a southward shift in mean rainband latitude has occurred during the Post-Meiyu (days 201-273, or July 20-Sep 30). Considering both primary and secondary rainbands north of 27\textdegree N, which are associated with the jet (Figure \ref{climo}d and Figure \ref{jet_seasonal}c), mean latitude during 1951-1979 was $33.6^\circ \textrm{N} \pm .3^\circ$ versus $32.9^\circ \textrm{N} \pm .3^\circ$ during 1980-2007 ($p=.0003$; Table S6). This shift remains significant if we do not restrict by front latitude ($p=.0048$). A Post-Meiyu rainfall increase in central China and decrease in northern China has also occurred, producing a South Flood-North Drought pattern (Figure \ref{changes_2d}c). As a result, yearly rainfall has increased in central China even though Pre-Meiyu rainfall changes in that region are negative (Figure \ref{changes_2d}a). Unlike \citet{Yu2010}, our catalog does not exhibit a \nth{20}-century decrease in the intensity of Yangtze River region frontal rainbands during July-August. A significant southward shift in rainband latitude is also found for the whole year ($p=.0032$, Table S6), but this signal is dominated by the Post-Meiyu shift.
	
\subsection{Changes in Rainband Attributes, 1994-2007 Versus 1979-1993}

	 In addition to the late 1970s change in China rainfall, earlier onset of rainfall over the South China Sea during 1994-2008 relative to 1979-1993 has been reported \citep{Kajikawa2012}, as well as an increase in rainfall over southern China and in the passage of tropical cyclones \citep{Kwon2007,Chang2014}. In Supplementary Tables S7 and S8, we find a rise in rainband intensity during Meiyu season (days 161-200) from 27.3 to 29.8 mm day$^{-1}$ ($p=.9994$), and a southward shift in rainband latitude from 30.0\textdegree N to 28.9\textdegree\ N ($p=.0002$). No significant changes are found in rainband frequency. The strengthening and southward shift of Meiyu season rainbands beginning in the mid-1990s merits further investigation. Our algorithm reveals its usefulness by isolating specific changes in rainband behavior that would not be distinguishable simply by looking at cumulative monthly rainfall.
			 				
\section{Conclusion}

	Using a recursive convergent image processing algorithm, we created an unprecedented database and 57-year climatology of frontal rainfall properties over China, including probability of rainband occurrence and mean latitude, intensity, tilt, width and length. Two statistically significant changes in rainband attributes occurred between the years 1951-1979 and 1980-2007: 1) A decrease in frequency during the Pre-Meiyu season (days 121-160, May 1-June 9; $p=.020$); and 2) A southward shift in latitude of rainbands during the Post-Meiyu season (days 201-273, July 20-Sep 30; $p=.0003$). The latter change is responsible for the South Flood-North Drought trend in total yearly rainfall. During 1994-2007 versus 1979-1993, a substantially different change is found ... Our algorithm calculates X \% of total rainfall falling over China from 1951 to 2007. The development of the rainband detection algorithm allows us to study both frontal rainfall, which is associated with larger-scale variability, and local storms, which may result from meso-scale features such as low mountains, and the distinct changes in each. Each type of rainfall is separately available our data set.
	 
	It is essential to understand whether the South Flood-North Drought will persist under \nth{21}-century warming, or manifests an ephemeral decadal change. However, the CMIP5 (Climate Model Intercomparison Project) model suite contained in the Intergovernmental Panel on Climate Change's Fifth Assessment Report (IPCC AR5) does not agree on the sign of future summer rainfall changes in East Asia \citep{Christensen2011}. In this study, we have found robust changes in frontal rainfall. The poleward expansion of the Hadley Cell is projected to continue under \nth{21}-century warming \citep{Lu2007,Kang2012}, but a recent study predicts that anomalous \nth{21}-century heating of the eastern Pacific Ocean will drive the Pacific jet further equatorward \citep{Park2014}. By linking the South Flood-North Drought to changes in the seasonal advance of the tropospheric jet, we open the possibility of projecting \nth{21}-century East Asian rainfall change by improving our understanding of the effect of further global warming on the regional and global behavior of the tropospheric jet.
	
	Many components of our results have been presented in previous work. \citet{Xuan2011} find a southward shift in the jet and increased Yangtze Valley rainfall in July. \citet{Yu2004} and \citet{Yu2007} found a southward shift in July-August jet latitude and suggested a link with the South Flood-North Drought. Potential mechanisms for late \nth{20}-century East Asian climate change include changes in Indian Ocean SST \citep{Qu2012}, decreased sensible heating from the Tibetan Plateau \citep{Liu2012a,Hu2015} and aerosol forcing \citep{Song2014}. Other studies attribute the South Flood-North Drought to natural variability \citep{Zhang1999,Xin2006,Lei2014}, but \citet{Zhou2009} claimed that the South Flood-North Drought was distinct from other patterns of \nth{20}-century variability. 


%%%  ACKNOWLEDGMENTS %%%

\begin{acknowledgments}
APHRODITE precipitation data is publicly available at \url{http://www.chikyu.ac.jp/precip/index.html}. Ferret, a NOAA product, was used for some data analysis and preliminary plot generation and is freely available at \url{http://ferret.pmel.noaa.gov/Ferret/}. The rainband detection algorithm and the majority of data analysis code were written in MATLAB. A full database of rainband statistics from 1 January 1951 to 31 December 2007 and associated MATLAB and Ferret codes used to produce results are available at the author's website: \url{http://www.atmos.berkeley.edu/~jessed/data.html}, and key figures are reproduced at \url{http://www.atmos.berkeley.edu/~jessed/myfigures.html}. This work was supported by NSF grants EAR-0909195 and EAR-1211925, which allowed the presentation of preliminary results in conference settings and the feedback of our peers. We also acknowledge NSFC (National Natural Science Foundation of China) grant \#40921120406 for enabling our collaboration with Professor Yanjun Cai of IEECAS in Xi'an, which led to the present work. We thank Jinqiang Chen and an anonymous reviewer for valuable suggestions for a prior version of the present manuscript. \end{acknowledgments}

%%% BIBLIOGRAPHY %%%

\bibliographystyle{agufull08}
\bibliography{meiyu}

%% ------------------------------------------------------------------------ %%
%
%  END ARTICLE
%
%% ------------------------------------------------------------------------ %%
\end{article}
%
%
%% Enter Figures and Tables here:
%
% DO NOT USE \psfrag or \subfigure commands.
%
% Figure captions go below the figure.
% Table titles go above tables; all other caption information
%  should be placed in footnotes below the table.
%


%%% Hovm�ller diagram of Meiyu latitude occupancy, 1951-2007. Produced by MATLAB scripts meiyufig1.m and meiyustats_compact.m.
\begin{figure}
\begin{center}
\noindent\includegraphics[width=30pc]{Figures/meiyu_hovmoller}
\caption{Hovm\"oller climatology of East Asian rainfall, 1951-2007, with important time periods marked as follows: 1 - Spring Rains; 2 - Pre-Meiyu; 3 - Meiyu; 4 - Post-Meiyu; 5 - Fall Rains. a) Precipitation averaged over the longitudes 100-123\textdegree E; b) Probability of occurrence of a rainband for each day and latitude (both primary and secondary, in percentage), smoothed in time with a 9-day running box filter; c) Probability of primary rainband occurrence and mean intensity (9-day running mean); d) The conditional probability of a secondary rainband given the presence of a primary rainband, as well as the mean tilt and length of primary rainband events (9-day running mean).}
\label{hov}
\end{center}
\end{figure}

%Climatology of rainfall stages including rainfall, jet and most likely rainband configuration, and longitudinal averages.
\begin{figure}
\noindent\includegraphics[width=36pc]{Figures/climo}
\caption{Climatology of East Asian rainfall stages showing rainfall (shading), jet kernel density (contours of probability density in units of $10^{-4}$) and most common rainband position during that stage. Sidebars shows, for each time period, the longitude average over 105-123$^{\circ}$E of rainfall (thin blue line, units of mm day$^{-1}$), jet kernel density (red line, units of $10^{-4}$) and rainband position (dashed black line, absolute probability in \%, 1-degree latitude smoothing). From the Pre-Meiyu to Post-Meiyu, a peak in preferred jet latitude consistently occurs 5 degrees north of a corresponding maximum in rainband frequency.}
\label{climo}
\end{figure}

%%% Changes in Meiyu and rainfall behavior between 1951-1979 and 1980-2007
\begin{figure}[htbp]
\begin{center}
\includegraphics[width=36pc]{Figures/changes}
\caption{a) 15-day running mean of the change in rainfall between 1951-1979 and 1980-07, with 95\%/99\% confidence level marked by single/double cross-hatches; b) 15-day running mean of the change in rainband frequency between 1951-1979 and 1980-07, with two-degree smoothing in latitude and confidence levels marked as in a). The significance of rainfall changes is calculated by a permutation method. Time periods are marked as in Figure 1.}
\label{changes}
\end{center}
\end{figure}

%2D spatial distribution of change showing a) full year b) Pre-Meiyu and c) Post-Meiyu
\begin{figure}
\noindent\includegraphics[width=36pc]{Figures/changes_2d}
\caption{a) Whole year mean rainfall change, showing the South Flood-North Drought pattern; b) Rainfall changes during the Pre-Meiyu (days 121-160) with contours of jet density change overlain; c) Same as c, but for the Post-Meiyu (days 201-273). Statistical significance at 95\%/99\% level overlain with single/double hatches. Sidebars show, for each time period, the longitude average over 105-123$^{\circ}$E of changes in rainfall (thin blue line, units of mm day$^{-1}$), jet kernel density (red line, units of $10^{-4}$) and rainband position (dashed black line, absolute probability in \%, 1-degree latitude smoothing).}
\label{changes_2d}
\end{figure}

\end{document}