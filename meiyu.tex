%%%%%%%%%%%%%%%%%%%%%%%%%%%%%%%%%%%%%%%%%%%%%%%%%%%%%%%%%%%%%%%%%%%%%%%%%%%%
% AGUtmpl.tex: this template file is for articles formatted with LaTeX2e,
% Modified March 2013
%
% This template includes commands and instructions
% given in the order necessary to produce a final output that will
% satisfy AGU requirements.
%
% PLEASE DO NOT USE YOUR OWN MACROS
% DO NOT USE \newcommand, \renewcommand, or \def.
%
% FOR FIGURES, DO NOT USE \psfrag or \subfigure.
%
%%%%%%%%%%%%%%%%%%%%%%%%%%%%%%%%%%%%%%%%%%%%%%%%%%%%%%%%%%%%%%%%%%%%%%%%%%%%
%
% All questions should be e-mailed to latex@agu.org.
%
%%%%%%%%%%%%%%%%%%%%%%%%%%%%%%%%%%%%%%%%%%%%%%%%%%%%%%%%%%%%%%%%%%%%%%%%%%%%
%
% Step 1: Set the \documentclass
%
% There are two options for article format: two column (default)
% and draft.
%
% PLEASE USE THE DRAFT OPTION TO SUBMIT YOUR PAPERS.
% The draft option produces double spaced output.
%
% Choose the journal abbreviation for the journal you are
% submitting to:

% jgrga JOURNAL OF GEOPHYSICAL RESEARCH
% gbc   GLOBAL BIOCHEMICAL CYCLES
% grl   GEOPHYSICAL RESEARCH LETTERS
% pal   PALEOCEANOGRAPHY
% ras   RADIO SCIENCE
% rog   REVIEWS OF GEOPHYSICS
% tec   TECTONICS
% wrr   WATER RESOURCES RESEARCH
% gc    GEOCHEMISTRY, GEOPHYSICS, GEOSYSTEMS
% sw    SPACE WEATHER
% ms    JAMES
% ef    EARTH'S FUTURE
%
%
%
% (If you are submitting to a journal other than jgrga,
% substitute the initials of the journal for "jgrga" below.)

\documentclass[draft,grl]{AGUTeX}
% To create numbered lines:

% If you don't already have lineno.sty, you can download it from
% http://www.ctan.org/tex-archive/macros/latex/contrib/ednotes/
% (or search the internet for lineno.sty ctan), available at TeX Archive Network (CTAN).
% Take care that you always use the latest version.

% To activate the commands, uncomment \usepackage{lineno}
% and \linenumbers*[1]command, below:

\usepackage{lineno}
\usepackage{amssymb}
\usepackage{amsmath}
\usepackage{textcomp}
\usepackage[super]{nth}
\usepackage{tabularx}
\usepackage{multirow}
\usepackage{bm}
\usepackage{color} 
\usepackage{hyperref}

\linenumbers*[1]

%  To add line numbers to lines with equations:
%  \begin{linenomath*}
%  \begin{equation}
%  \end{equation}
%  \end{linenomath*}
%%%%%%%%%%%%%%%%%%%%%%%%%%%%%%%%%%%%%%%%%%%%%%%%%%%%%%%%%%%%%%%%%%%%%%%%%
% Figures and Tables
%
%
% DO NOT USE \psfrag or \subfigure commands.
%
%  Figures and tables should be placed AT THE END OF THE ARTICLE,
%  after the references.
%
%  Uncomment the following command to include .eps files
%  (comment out this line for draft format):
\usepackage[final]{graphicx}
%
%  Uncomment the following command to allow illustrations to print
%   when using Draft:
%  \setkeys{Gin}{draft=false}
%
% Substitute one of the following for [dvips] above
% if you are using a different driver program and want to
% proof your illustrations on your machine:
%
% [xdvi], [dvipdf], [dvipsone], [dviwindo], [emtex], [dviwin],
% [pctexps],  [pctexwin],  [pctexhp],  [pctex32], [truetex], [tcidvi],
% [oztex], [textures]
%
% See how to enter figures and tables at the end of the article, after
% references.
%
%% ------------------------------------------------------------------------ %%
%
%  ENTER PREAMBLE
%
%% ------------------------------------------------------------------------ %%

% Author names in capital letters:
\authorrunninghead{DAY ET AL.}

% Shorter version of title entered in capital letters:
\titlerunninghead{THE ``SOUTH FLOOD-NORTH DROUGHT'' IN MEIYU AND JET STATS}

%Corresponding author mailing address and e-mail address:
\authoraddr{Corresponding author: Jesse Day, University of California Berkeley, Department of Earth and Planetary Science, College of Letters and Science; 307 McCone Hall, Berkeley, CA 94720, USA. (jessed@berkeley.edu)}

\begin{document}

%% ------------------------------------------------------------------------ %%
%
%  TITLE
%
%% ------------------------------------------------------------------------ %%


\title{Signature of the ``South Flood-North Drought" in Meiyu Front and Tropospheric Jet Changes}

%% ------------------------------------------------------------------------ %%
%
%  AUTHORS AND AFFILIATIONS
%
%% ------------------------------------------------------------------------ %%


%Use \author{\altaffilmark{}} and \altaffiltext{}

% \altaffilmark will produce footnote;
% matching \altaffiltext will appear at bottom of page.

\authors{Jesse A. Day\altaffilmark{1},
Jacob P. Edman\altaffilmark{1}, John C. H. Chiang\altaffilmark{2}, Inez Fung \altaffilmark{1}, and
Weihan Liu\altaffilmark{3}}

\altaffiltext{1}{Department of Earth and Planetary Science, University of California Berkeley, Berkeley, California, USA.}
\altaffiltext{2}{Department of Geography, University of California Berkeley, Berkeley, California, USA.}
\altaffiltext{3}{College of Letters and Science, University of California Berkeley, Berkeley, California, USA.}


%% ------------------------------------------------------------------------ %%
%
%  ABSTRACT
%
%% ------------------------------------------------------------------------ %%

% >> Do NOT include any \begin...\end commands within
% >> the body of the abstract.

%Needs to be 150 words or less - currently at 147.
\begin{abstract}
A novel 57-year (1951-2007) daily catalog of frontal rainbands over China is compiled based on APHRODITE rain gauge data, resulting in an unprecedented climatology of Meiyu front progression in summer. Late \nth{20}-century changes in Chinese summer rainfall are investigated (the ``South Flood-North Drought''). Two robust changes in front behavior are observed during 1980-2007 relative to 1951-1979: 1) A significant decrease in the frequency of frontal rainbands during the Pre-Meiyu period (May), and 2) a southward shift in Post-Meiyu rainbands (mid-July to September). In addition, the mean latitude of the tropospheric jet has shifted southward during both time periods. We propose that the observed changes in frontal rainfall reflect an overall southward displacement of the tropospheric jet's summer progression. By linking East Asian climate change to larger-scale change, our results begin to address the critical question of whether the South Flood-North Drought will persist under \nth{21}-century global warming.
\end{abstract}

%% Allowed length of manuscript is (# of words/500) + # of figures + # of tables.

%currently: 3489 words + 5 figures (max words: 3500)

\begin{article}

\section{Introduction}
 
 	Eastern China receives about 60\% of its rainfall from May to August via the East Asian summer monsoon. The period of peak rainfall lasting from early June to mid-July is called ``Meiyu season'' (lit. ``plum rains,'' referring to the spectacular growth of plum blossoms in central China with the onset of heavy rains). During this time, heavy rainfall occurs in zonal bands resulting from frontal synoptic conditions (the ``Meiyu front''). The rainfall climatology of Japan and Korea also features similar phenomena known as Baiu and Changma respectively. A growing volume of evidence suggests a shift in rainfall over China beginning in the late 1970s, featuring a ``South Flood-North Drought'' pattern shown in Figure \ref{changes_2d}a \citep{Hu1997,Gong2002,Nigam2013}. Due to the severe human impacts of the South Flood-North Drought on densely-populated eastern China, it is vital to understand whether this pattern will strengthen under global warming, or represents only a temporary deviation from the mean.
 
	The climatology of the East Asian monsoon bears little resemblance to other monsoon circulations \citep{Ding2005}. Whereas understanding of tropical monsoons has progressed greatly via theoretical studies \citep{Plumb1992,Prive2007,Bordoni2008}, the dynamics that favor the existence of frontal convection over East Asia in summer remain a point of debate, centering around the interplay of the tropospheric jet and Tibetan Plateau \citep{Molnar2010,Sampe2010,Chen2014}. Therefore, no simple conceptual template exists for interpreting a change such as the South Flood-North Drought. However, it is known that the migration of the Meiyu front entails a series of large-scale circulation changes \citep{Chen2004}, and furthermore that anomalies in Meiyu front latitude produce corresponding rainfall anomalies \citep{Kosaka2011}. Therefore, the South Flood-North Drought should be describable in terms of changes in the properties of Meiyu rainbands, such as a shift in latitude, a change in intensity or an earlier or delayed northward migration. In turn, such a characterization may provide insight into the dynamics responsible for the change.
	
	In pursuit of this aim, we have developed a 57-year database (1951-2007) of frontal rainbands in China based on the APHRODITE rain gauge product (described below). We develop a recursive convergent fitting algorithm of daily rainfall maps which finds rainbands and quantifies their attributes. Previous studies have investigated the statistics of the Meiyu front on decadal and even centennial timescales \citep{Chen2004,Ge2008,Xu2009}, but to our knowledge no previous author has compiled a multi-decadal daily catalog of events. We use this catalog to clarify the spatial and temporal attributes of the South Flood-North Drought, and present it as a tool for future East Asian monsoon research.
		
	We also explore decadal changes in the subtropical jet, which plays an essential and complex role in East Asian climate both in summer and winter \citep{Molnar2010,Yang2002}. In a region of strong fronts as observed in East Asia, theory predicts that the core of maximum zonal wind should anchor an equatorward region of ascent and strong rainfall \citep{Holton2004}. The Tibetan Plateau couples with the jet nonlinearly, amplifying the regional response to global climate anomalies \citep{Nigam1989,Broccoli1992,Park1997}. The jet's passage north of the Tibetan Plateau in summer is argued to dictate the timing of onset of the Indian and East Asian monsoons, both in present-day \citep{Yin1949,Yeh1959,Hahn1975} and on paleoclimate timescales \citep{Nagashima2011,Nagashima2013,Chiang2015}. On a weekly timescale, the jet serves as a waveguide for storms propagating from the Eurasian interior via the ``Silk Road'' teleconnection \citep{Hoskins1993,Ambrizzi1997,Kosaka2012}, and shifts in its latitude and strength induce corresponding rainfall anomalies \citep{Liang1998,Kwon2007,Du2009,Li2014}. Thus, we expect that the South Flood-North Drought has also entailed \nth{20}-century changes in the East Asian tropospheric jet. Therefore, we compare our rainband database to a database of jet counts from 1958 to 2001 from \citet{Schiemann2009} in search of coupled change.
	
\section{Data and Methods}

\subsection{APHRODITE}

	The APHRO\_MA\_V1101 product from APHRODITE (Asian Precipitation - Highly-Resolved Observational Data Integration Towards Evaluation of the Water Resources) includes 57 years (1951-2007) of daily rainfall (PRECIP product) on a .25\textdegree\ $\times$ .25\textdegree\ grid over 60-150\textdegree E and 15\textdegree S-55\textdegree N \citep{Yatagai2012}. Values are assimilated from weather station observations and therefore available over land only. We focus on the subregion inside of 100\textdegree E-123\textdegree E and 20\textdegree N-40\textdegree N, where Meiyu rainbands occur. Stations in this region are spaced at 100-200 km intervals (shown by RSTN product), such that rainbands are clearly resolved. APHRODITE's resolution cannot capture some features visible in TRMM satellite data \citep{Xu2009}, but its length allows for the study of decadal change.
	
\subsection{Jet Count Density} 

	\citet{Schiemann2009} constructed a data set of jet `counts' in the Tibetan Plateau region (46\textdegree E-130\textdegree E, 17\textdegree N-58\textdegree N) from ERA-40 reanalysis for 1958-2001, where a count is defined as any local maximum in zonal wind with westerly magnitude greater than $30$ m s$^{-1}$; further details can be found in section 2 of \citet{Schiemann2009}. We show daily mean jet latitude averaged across $90-130^\circ$E in Figure \ref{jet_seasonal}a and monthly anomalies in Figure \ref{jet_seasonal}b. Results are not sensitive to the choice of longitude range. Figure \ref{climo} presents contours of jet frequency estimated by a kernel density method, which estimates a probability distribution from a set of discrete data observations (Supplementary Text S4).
	
\subsection{Rainband Detection Algorithm}

	For each day from 1 January 1951 to 31 December 2007 (20,819 days total), our algorithm determines whether a rainband exists inside the window of 105-123\textdegree E and 20-40\textdegree N. A rainband is defined as a continuous chain of rainfall maxima exceeding 10 mm day$^{-1}$ spanning more than 5\textdegree of longitude. If such a band exists, a recursive algorithm finds its position. Subsequently, properties of the rainband are calculated including latitude, intensity, tilt, length and width, as well as a ``quality score'' $Q$, defined as the fraction of daily rainfall occurring within the band. Days where $Q<.6$ are thrown out. We also test for the existence of two rainbands on a single day, an arrangement commonly found in August and September. In such a case, the first and second fitted rainbands are referred to as ``primary'' and ``secondary'' rainbands respectively. Days dominated by heavy local rainfall over Taiwan are discarded. A step-by-step description of the algorithm is presented in Supplementary Text S2 and Supplementary Figures S1-S4, and metrics of its performance documented in Supplementary Tables S1-S3. Our algorithm does not distinguish between the mechanisms that supply rainfall.

\subsection{Bootstrapping and Significance of Changes}

	The standard deviation and significance of changes in rainband frequency are calculated analytically. However, rainband statistics and jet count density do not follow a normal distribution. Therefore, we use bootstrapping with replacement to calculate the standard deviation of their means (Figure \ref{jet_seasonal}a and Supplementary Tables S4-S8 respectively). To calculate the statistical significance of differences in mean between two samples, two methods are used: 1) Bootstrapping with replacement and 2) a permutation test (bootstrapping without replacement) \citep{Good2005}. Both produce similar results; $p$-values shown are from permutation testing. Figure 3a uses a moving blocks bootstrap with block length of 3 days (Supplementary Text S3). We focus on changes in front attributes between 1951-1979 and 1980-2007 (Supplementary Tables S5 and S6), and successfully verified that results are robust using only the years spanned by the jet database (1980-2001 versus 1958-1979; not shown).
	
\section{Rainband Statistics}	
	
\subsection{Climatology}	

	The yearly progression of precipitation over eastern China is shown in Figure \ref{hov}a, longitudinally averaged over $100-123^\circ$E with a 5-day running mean, similar to Figure 7 in \citet{Ding2005}. China receives a substantial fraction of its yearly precipitation outside of summer, unlike other monsoonal regions \citep{Wang2002}. Figure \ref{hov}b shows a Hovm\"oller diagram of rainband frequency over all 57 years, including both primary and secondary rainbands. Some periods of heavy rainfall, in particular the August peak over southern China (over 10 mm day$^{-1}$ around 20\textdegree N), do not correspond to a surge in rainbands. Figure \ref{hov}c shows the probability of observing a rainband and mean rainband intensity, and Figure \ref{hov}d shows mean rainband tilt and length, as well as the conditional probability of observing a secondary rainband given the presence of a primary rainband. Frontal rainbands over China can appear in any month, with their probability of occurrence and intensity maximizing in late June (80\% probability of occurrence, mean intensity of 31 mm day$^{-1}$) and minimizing in January (10\% probability occurrence, mean intensity of 12 mm day$^{-1}$).
	
	Coordinated, abrupt changes occur in rainfall and frontal climatology. We define 5 periods of notable behavior as demarcated in Figure \ref{hov}: 1) The ``Spring Rains'' (days 60-120, March 1-April 30), as previously studied in \citet{Tian1998}; 2) The ``Pre-Meiyu'' (121-160, May 1-June 9), during which rainfall and front intensity increase; 3) Meiyu season (161-200, June 10-July 19) when a remarkable 7-degree northward shift in mean rainband latitude occurs over the course of several weeks, and rainband frequency and intensity peaks; 4) The Post-Meiyu (201-273, July 20-September 30), during which double rainbands are common; and 5) the ``Fall Rains'' (274-320, October 1-November 16), when rainband latitude returns south. The Pre-Meiyu, Meiyu and Post-Meiyu are equivalent to the three stages of Meiyu rainfall described in \citet{Ding2005}. Our results can also  be compared with the event catalog of \citet{Xu2009}, which finds a similar date for the northward transition of the Meiyu front. The total number of rainband counts as well as the mean and standard deviation of rainband frequency, latitude and intensity during each time period are presented in Supplementary Table S4.  In addition, Figures \ref{climo}a-e show mean rainfall, jet frequency and rainband position during each stage, as well as their zonal average (sidebars). From the Pre-Meiyu to Post-Meiyu, each northward jump in peak rainband frequency corresponds to a similar shift in jet count density, with a southward offset of about 5 degrees. A yearly asymmetry can also be seen between more frequent and intense rainbands during the jet's northward passage (Pre-Meiyu and Meiyu) versus weaker rainfall during its southward return (Fall Rains), which merits further study.
		
\subsection{Changes in Rainband Attributes, 1980-2007 Versus 1951-1979}
	
	We calculate changes in rainfall and rainband frequency during 1980-2007 relative to 1951-1979, along with their statistical significance (Figure \ref{changes}). In addition, we evaluate the significance of changes in rainband attributes between these sets of years during each of the five rainfall stages (Supplementary Tables S5 and S6). Finally, Figures \ref{changes_2d}b and \ref{changes_2d}c show spatial changes in rainfall and jet count density during the Pre-Meiyu and Post-Meiyu, when changes are particularly large. During the Pre-Meiyu (days 121-160), the probability of observing a primary rainband has declined from $59.0\% \pm 2.0\%$ to $53.0\% \pm 2.1\%$ ($p=0.020$; Table S5). A corresponding decrease in Pre-Meiyu rainfall has occurred in central China (Figure \ref{changes_2d}b and 30\textdegree N in Figure \ref{changes}a). The change in Pre-Meiyu rainfall in the late \nth{20} century has previously been reported by \citet{Xin2006} and \citet{Wang2009}.
		
	In addition, a southward shift in mean rainband latitude has occurred during the Post-Meiyu (days 201-273, or July 20-Sep 30). Considering both primary and secondary rainbands north of 27\textdegree N, which are associated with the jet (Figure \ref{climo}d and Figure \ref{jet_seasonal}c), mean latitude during 1951-1979 was $33.6^\circ \textrm{N} \pm .3^\circ$ versus $32.9^\circ \textrm{N} \pm .3^\circ$ during 1980-2007 ($p=.0003$; Table S6). This shift remains significant if we do not restrict by front latitude ($p=.0048$). A Post-Meiyu rainfall increase in central China and decrease in northern China has also occurred, producing a South Flood-North Drought pattern (Figure \ref{changes_2d}c). As a result, yearly rainfall has increased in central China even though Pre-Meiyu rainfall changes in that region are negative (Figure \ref{changes_2d}a). Unlike \citet{Yu2010}, our catalog does not exhibit a \nth{20}-century decrease in the intensity of Yangtze River region frontal rainbands during July-August. A significant southward shift in rainband latitude is also found for the whole year ($p=.0032$, Table S6), but this signal is dominated by the Post-Meiyu shift.
	
\section{Jet}

\subsection{Dynamics}

	The tropospheric jet marks the northern boundary of the Hadley Cell, and shifts in response to seasonal changes in insolation \citep{Bordoni2008}. Beginning in May, the East Asian jet moves from its winter position on the southern flank of the Tibetan Plateau to a summer latitude well north of the plateau. During this transition, the jet occupies intermediate configurations that correspond to different stages of China rainfall (Figure \ref{climo}). A full monthly jet climatology is visible in \citet{Schiemann2009}. Peak rainfall rates in China from May to mid-July corresponds to the months when the climatological latitude of the jet impinges on the Tibetan Plateau. The interaction of the tropospheric jet and Tibetan Plateau strengthens convergence and rainfall downstream over China and the western Pacific Ocean \citep{Molnar2010,Sampe2010,Chen2014}. From May to September,  the climatological latitude of rainfall, rainbands and jet density are all closely coupled, with peak jet density occurring 5 to 10 degrees north of the latitude of peak rainfall. The initiation of the Pre-Meiyu corresponds roughly to the beginning of the jet's northward passage. During Meiyu season, the preferred latitude of the jet continues to shift northward. The period of frequent double rainband occurrence during the Post-Meiyu corresponds to the jet's maximal northward extent. Finally, the jet returns southward during the Fall Rains in October and November, which produces only a weak rainfall response.
	
\subsection{Jet Changes, 1980-2001 Versus 1958-1979}

	In Figure \ref{jet_seasonal}a, we show the zonal average over $90-130^\circ$E) of mean jet latitude, averaged over the years 1958-1979 (blue solid line) and 1980-2001 (dashed red line) with 95\% confidence intervals overlain. Both significant changes in rainband statistics described in the previous section correspond to southward shifts in mean jet latitude. During the Pre-Meiyu (May), the tropospheric jet is shifted southward by almost 2\textdegree\ in 1980-2001 relative to 1958-1979, when its mean latitude was $\approx 41^\circ$N. We estimate the significance of this change using a two-tailed Kolmogorov-Smirnov (K-S) test. Since the K-S test requires that all samples are independent, we first remove temporal autocorrelation due to synoptic variability by assimilating daily mean jet latitude into 4 day blocks (Supplementary Figure S5). A subsequent K-S test finds that the shift is significant with $p=0.003998$. During the Post-Meiyu (days 201-273), when a southward shift in rainband latitude is found in 1980-2001 relative to 1958-1979, the mean latitude of the jet is also consistently displaced southward. We assimilate daily mean jet latitude into 7-day blocks (Supplementary Figure S5) before performing a K-S test, and find a $p$-value of this shift of $p=0.05667$.
		
\section{Discussion}

	The Meiyu front and tropospheric jet covary in latitude from May to September in the climatological mean, and parallel changes are found in rainband attributes and mean jet latitude between 1951-1979 and 1980-2007. Therefore, the South Flood-North Drought appears to reflect an alteration in jet dynamics. We propose that both the Pre-Meiyu decline in rainband frequency and the Post-Meiyu southward shift of rainband latitude result from a single phenomenon: an overall southward displacement of the jet's summer progression over East Asia. In climatology, the Pre-Meiyu corresponds to both a surge in rainfall and the beginning of the jet's northward transit, when its preferred latitude begins to impinge on the Tibetan Plateau. We propose that the observed southward shift in the jet during May has delayed the date when the jet first impinges on the Tibetan Plateau, resulting in a delay in Pre-Meiyu onset and prolonged Spring Rain conditions. This is manifested as weaker rainfall and decreased rainband frequency in central China in May. Subsequently, we argue that the reduced northward extent of the jet during the Post-Meiyu has shifted rainfall and mean rainband latitude southward. Finally, we suggest that the southward displacement of the summer jet cycle results in a decrease in northern China annual rainfall and an increase in central China annual rainfall, producing a South Flood-North Drought response. Thus, our hypothesis can explain the major observed changes in rainfall and rainband statistics during the Pre- and Post-Meiyu as well as cumulative yearly change.
	
	To test our hypothesis, we investigate the relation of interannual anomalies in jet latitude and rainband properties. Figure \ref{jet_seasonal}b shows a scatter plot of rainband \textit{frequency} anomalies versus jet latitude anomalies in May (days 121-150). Most years with a decrease in rainband frequency feature a southward jet shift, and vice-versa, and such years occur mostly during 1980-2007. A similar relation is found between monthly anomalies in rainband \textit{latitude} and jet latitude during July-August (days 201-260, Figure \ref{jet_seasonal}c). In the latter figure, we exclude rainbands south of 28\textdegree N from calculated anomalies, since such rainbands reflect South China Sea storms, rather than jet influence \citep{Day2015}. Together, Figures \ref{jet_seasonal}b and \ref{jet_seasonal}c suggest that interannual changes in jet latitude affect Pre-Meiyu rainband frequency and Post-Meiyu rainband latitude.

	 Observations show that the global annual mean latitude of the tropospheric jet has shifted poleward, in tandem with tropospheric heating and lower-stratospheric cooling in the mid-latitudes, increased subtropical static stability, and the expansion of the Hadley circulation \citep{Fu2006,Archer2008,Fu2011}. Opposite trends are found in some regions and the variation by season is significant; we find that the East Asian portion of the jet has shifted equatorward, in agreement with past studies \citep{Yu2007, Archer2008}. Recent work proposes that the observed southward displacement of the jet over the Pacific Ocean was caused by \nth{20}-century changes in tropical Pacific convection and SST \citep{Park2014a}. Thus, the global poleward trend in jet latitude and the East Asian equatorward shift are compatible observations that reflect the heterogeneous spatial distribution of \nth{20}-century warming.
	 
	 In addition to the late 1970s change in China rainfall, earlier onset of rainfall over the South China Sea during 1994-2008 relative to 1979-1993 has been reported \citep{Kajikawa2012}, as well as an increase in rainfall over southern China and in the passage of tropical cyclones \citep{Kwon2007,Chang2014}. In Supplementary Tables S7 and S8, we find a rise in rainband intensity during Meiyu season (days 161-200) from 27.3 to 29.8 mm day$^{-1}$ ($p=.9994$), and a southward shift in rainband latitude from 30.0\textdegree N to 28.9\textdegree\ N ($p=.0002$). No significant changes are found in rainband frequency or jet latitude. The strengthening and southward shift of Meiyu season rainbands beginning in the mid-1990s merits further investigation.
		
\section{Conclusion}

	We have shown that a significant amount of the annual and decadal variability in Meiyu front activity is accompanied by changes in the westerly jet. Using a recursive convergent image processing algorithm, we created an unprecedented database and 57-year climatology of frontal rainfall properties over China, including probability of rainband occurrence and mean latitude, intensity, tilt, width and length. Two statistically significant changes in rainband attributes occurred between the years 1951-1979 and 1980-2007: 1) A decrease in frequency during the Pre-Meiyu season (days 121-160, May 1-June 9; $p=.020$); and 2) A southward shift in latitude of rainbands during the Post-Meiyu season (days 201-273, July 20-Sep 30; $p=.0003$). The latter change is responsible for the South Flood-North Drought trend in total yearly rainfall. In addition, both time periods display a southward anomaly in mean jet latitude during 1980-2007 relative to 1951-1979. We argue that both Pre-Meiyu and Post-Meiyu changes in rainfall and rainband statistics are caused by a southward shift of the summer progression of the East Asian tropospheric jet. In particular, we propose that the delayed passage of the jet to the north of the Tibetan Plateau has shortened the Pre-Meiyu season, decreasing May rainfall in central China, and restricted the northward advance of precipitation, consequently reducing Post-Meiyu rainfall in northern China. This interpretation is a modern analog of the ``Jet Transition Hypothesis'' described in \citet{Chiang2015}, wherein East Asian rainfall changes on paleoclimate timescales are ascribed to modulation in the seasonal cycle of the tropospheric jet. 	
 
	Many components of our results have been presented in previous work. \citet{Xuan2011} find a southward shift in the jet and increased Yangtze Valley rainfall in July. \citet{Yu2004} and \citet{Yu2007} found a southward shift in July-August jet latitude and suggested a link with the South Flood-North Drought. Potential mechanisms for late \nth{20}-century East Asian climate change include changes in Indian Ocean SST \citep{Qu2012}, decreased sensible heating from the Tibetan Plateau \citep{Liu2012a,Hu2015} and aerosol forcing \citep{Song2014}. Other studies attribute the South Flood-North Drought to natural variability \citep{Zhang1999,Xin2006,Lei2014}, but \citet{Zhou2009} claimed that the South Flood-North Drought was distinct from other patterns of \nth{20}-century variability. Our hypothesis, that the altered seasonal cycle of the jet induces observed rainfall change, does not supplant prior explanations but rather provides further observations that they must address.
	 
	It is essential to understand whether the South Flood-North Drought will persist under \nth{21}-century warming, or manifests an ephemeral decadal change. However, the CMIP5 (Climate Model Intercomparison Project) model suite contained in the Intergovernmental Panel on Climate Change's Fifth Assessment Report (IPCC AR5) does not agree on the sign of future summer rainfall changes in East Asia \citep{Christensen2011}. In this study, we have tied the regional rainfall climate of China to the seasonal progression of the westerly jet, which is a larger-scale feature and therefore more easily studied in global climate models and idealized studies. The poleward expansion of the Hadley Cell is projected to continue under \nth{21}-century warming \citep{Lu2007,Kang2012}, but a recent study predicts that anomalous \nth{21}-century heating of the eastern Pacific Ocean will drive the Pacific jet further equatorward \citep{Park2014}. By linking the South Flood-North Drought to changes in the seasonal advance of the tropospheric jet, we open the possibility of projecting \nth{21}-century East Asian rainfall change by improving our understanding of the effect of further global warming on the regional and global behavior of the tropospheric jet.
	

%%%  ACKNOWLEDGMENTS %%%

\begin{acknowledgments}
We thank Reinhardt Schiemann for sharing his database of jet counts. APHRODITE precipitation data is publicly available at \url{http://www.chikyu.ac.jp/precip/index.html}. Ferret, a NOAA product, was used for some data analysis and preliminary plot generation and is freely available at \url{http://ferret.pmel.noaa.gov/Ferret/}. The rainband detection algorithm and the majority of data analysis code were written in MATLAB. A full database of rainband statistics from 1 January 1951 to 31 December 2007 and associated MATLAB and Ferret codes used to produce results are available at the author's website: \url{http://www.atmos.berkeley.edu/~jessed/data.html}, and key figures are reproduced at \url{http://www.atmos.berkeley.edu/~jessed/myfigures.html}. This work was supported by NSF grants AGS-1405479, EAR-0909195 and EAR-1211925, which allowed the presentation of preliminary results in conference settings and the feedback of our peers. We also acknowledge NSFC (National Natural Science Foundation of China) grant \#40921120406 for enabling our collaboration with Professor Yanjun Cai of IEECAS in Xi'an, which led to the present work. We thank Gerard Roe for a valuable comment on the poleward global shift of jet latitude. 
\end{acknowledgments}

%%% BIBLIOGRAPHY %%%

%\bibliographystyle{agufull08}
%\bibliography{meiyu}

\begin{thebibliography}{61}
\providecommand{\natexlab}[1]{#1}
\expandafter\ifx\csname urlstyle\endcsname\relax
  \providecommand{\doi}[1]{doi:\discretionary{}{}{}#1}\else
  \providecommand{\doi}{doi:\discretionary{}{}{}\begingroup
  \urlstyle{rm}\Url}\fi

\bibitem[{\textit{Ambrizzi and Hoskins}(1997)}]{Ambrizzi1997}
Ambrizzi, T., and B.~J. Hoskins (1997), {Stationary Rossby-��wave propagation
  in a baroclinic atmosphere}, \textit{Q. J. R. Meteorol. Soc.}, \textit{123},
  919--928.

\bibitem[{\textit{Archer and Caldeira}(2008)}]{Archer2008}
Archer, C.~L., and K.~Caldeira (2008), {Historical trends in the jet streams},
  \textit{Geophys. Res. Lett.}, \textit{35}, L08,803,
  \doi{10.1029/2008GL033614}.

\bibitem[{\textit{Bordoni and Schneider}(2008)}]{Bordoni2008}
Bordoni, S., and T.~Schneider (2008), {Monsoons as eddy-mediated regime
  transitions of the tropical overturning circulation}, \textit{Nat. Geosci.},
  \textit{1}, 515--519, \doi{10.1038/ngeo248}.

\bibitem[{\textit{Broccoli and Manabe}(1992)}]{Broccoli1992}
Broccoli, A.~J., and S.~Manabe (1992), {The Effects of Orography on Midlatitude
  Northern Hemisphere Dry Climates}, \textit{J. Clim.}, \textit{5}, 1181--1201.

\bibitem[{\textit{Chang et~al.}(2014)\textit{Chang, Yeh, Hong, Kim, Wu, and
  Kei}}]{Chang2014}
Chang, E.-C., S.-W. Yeh, S.-Y. Hong, J.-E. Kim, R.~Wu, and Y.~Kei (2014),
  {Study on the changes in the East Asian precipitation in the mid�-1990s using
  a high�-resolution global downscaled atmospheric data set}, \textit{J.
  Geophys. Res. Atmos.}, \textit{119}, 2279--2293, \doi{10.1002/2013JD020903}.

\bibitem[{\textit{Chen}(2004)}]{Chen2004}
Chen, G. T.-J. (2004), {Research on the phenomena of Meiyu during the past
  quarter century: An overview}, in \textit{East Asian Monsoon}, edited by
  C.-P. Chang, pp. 357--403, World Scientific, Singapore.

\bibitem[{\textit{Chen and Bordoni}(2014)}]{Chen2014}
Chen, J., and S.~Bordoni (2014), {Orographic Effects of the Tibetan Plateau on
  the East Asian Summer Monsoon: An Energetic Perspective}, \textit{J. Clim.},
  \textit{27}, 3052--3072, \doi{10.1175/JCLI-D-13-00479.1}.

\bibitem[{\textit{Chiang et~al.}(2015)\textit{Chiang, Fung, Wu, Cai, Edman,
  Liu, Day, Bhattacharya, Mondal, and Labrousse}}]{Chiang2015}
Chiang, J.~C., I.~Y. Fung, C.-H. Wu, Y.~Cai, J.~P. Edman, Y.~Liu, J.~A. Day,
  T.~Bhattacharya, Y.~Mondal, and C.~A. Labrousse (2015), {Role of seasonal
  transitions and westerly jets in East Asian paleoclimate}, \textit{Quat. Sci.
  Rev.}, \textit{108}, 111--129, \doi{10.1016/j.quascirev.2014.11.009}.

\bibitem[{\textit{Christensen et~al.}(2011)\textit{Christensen, {Krishna
  Kumar}, Aldrian, An, Cavalcanti, de~Castro, Dong, Goswami, Hall, Kanyanga,
  Kitoh, Kossin, Lau, Renwick, Stephenson, Xie, and Zhou}}]{Christensen2011}
Christensen, J.~H., K.~{Krishna Kumar}, E.~Aldrian, S.-I. An, I.~F.~A.
  Cavalcanti, M.~de~Castro, W.~Dong, P.~Goswami, A.~Hall, J.~K. Kanyanga,
  A.~Kitoh, J.~Kossin, N.-C. Lau, J.~Renwick, D.~B. Stephenson, S.-P. Xie, and
  T.~Zhou (2011), {Climate Phenomena and their Relevance for Future Regional
  Climate Change}, in \textit{Climate Change 2013: The Physical Science Basis.
  Contribution of Working Group I to the Fifth Assessment Report of the
  Intergovernmental Panel on Climate Change}, edited by T.~F. Stocker, D.~Qin,
  G.-K. Plattner, M.~Tignor, S.~K. Allen, J.~Boschung, A.~Nauels, Y.~Xia,
  V.~Bex, and P.~M. Midgley, Cambridge University Press, Cambridge, United
  Kingdom and New York, NY, USA.

\bibitem[{\textit{Day et~al.}(2015)\textit{Day, Fung, and Risi}}]{Day2015}
Day, J.~A., I.~Fung, and C.~Risi (2015), {Coupling of South and East Asian
  Monsoon Precipitation in July-August}, \textit{J. Clim.},
  \doi{10.1175/JCLI-D-14-00393.1}.

\bibitem[{\textit{Ding and Chan}(2005)}]{Ding2005}
Ding, Y., and J.~C.~L. Chan (2005), {The East Asian summer monsoon: an
  overview}, \textit{Meteorol. Atmos. Phys.}, \textit{89}, 117--142,
  \doi{10.1007/s00703-005-0125-z}.

\bibitem[{\textit{Du et~al.}(2009)\textit{Du, Zhang, and Xie}}]{Du2009}
Du, Y., Y.~Zhang, and Z.~Xie (2009), {Impacts of the Zonal Position of the East
  Asian Westerly Jet Core on Precipitation Distribution During Meiyu of China},
  \textit{Acta Meteorol. Sin.}, \textit{23}, 506--516.

\bibitem[{\textit{Fu and Lin}(2011)}]{Fu2011}
Fu, Q., and P.~Lin (2011), {Poleward shift of subtropical jets inferred from
  satellite-observed lower-stratospheric temperatures}, \textit{J. Clim.},
  \textit{24}, 5597--5603, \doi{10.1175/JCLI-D-11-00027.1}.

\bibitem[{\textit{Fu et~al.}(2006)\textit{Fu, Johanson, Wallace, and
  Reichler}}]{Fu2006}
Fu, Q., C.~Johanson, J.~Wallace, and T.~Reichler (2006), {Enhanced mid-latitude
  tropospheric warming in satellite measurements}, \textit{Science},
  \textit{312}, 1179.

\bibitem[{\textit{Ge et~al.}(2008)\textit{Ge, Guo, Zheng, and Hao}}]{Ge2008}
Ge, Q., X.~Guo, J.~Zheng, and Z.~Hao (2008), {Meiyu in the middle and lower
  reaches of the Yangtze River since 1736}, \textit{Chinese Sci. Bull.},
  \textit{53}, 107--114, \doi{10.1007/s11434-007-0440-5}.

\bibitem[{\textit{Gong and Ho}(2002)}]{Gong2002}
Gong, D.-Y., and C.-H. Ho (2002), {Shift in the summer rainfall over the
  Yangtze River valley in the late 1970s}, \textit{Geophys. Res. Lett.},
  \textit{29}, 1436, \doi{10.1029/2001GL014523}.

\bibitem[{\textit{Good}(2005)}]{Good2005}
Good, P. (2005), \textit{{Permutation, parametric and bootstrap tests of
  hypotheses}}, 3rd ed., Springer, New York, NY.

\bibitem[{\textit{Hahn and Manabe}(1975)}]{Hahn1975}
Hahn, D.~G., and S.~Manabe (1975), {The role of mountains in the south Asian
  monsoon circulation}, \textit{J. Atmos. Sci.}, \textit{32}.

\bibitem[{\textit{Holton}(2004)}]{Holton2004}
Holton, J.~R. (2004), \textit{{An Introduction to Dynamic Meteorology}}, 4th
  ed., 269--278 pp., Academic Press.

\bibitem[{\textit{Hoskins and Ambrizzi}(1993)}]{Hoskins1993}
Hoskins, B.~J., and T.~Ambrizzi (1993), {Rossby Wave Propagation on a Realistic
  Longitudinally Varying Flow}, \textit{J. Atmos. Sci.}, \textit{50},
  1661--1671.

\bibitem[{\textit{Hu and Duan}(2015)}]{Hu2015}
Hu, J., and A.~Duan (2015), {Relative contributions of the Tibetan Plateau
  thermal forcing and the Indian Ocean Sea surface temperature basin mode to
  the interannual variability of the East Asian summer monsoon}, \textit{Clim.
  Dyn.}, \doi{10.1007/s00382-015-2503-7}.

\bibitem[{\textit{Hu}(1997)}]{Hu1997}
Hu, Z.-Z. (1997), {Interdecadal variability of summer climate over East Asia
  and its association with 500 hPa height and global sea surface temperature},
  \textit{J. Geophys. Res.}, \textit{102}, 19,403, \doi{10.1029/97JD01052}.

\bibitem[{\textit{Kajikawa and Wang}(2012)}]{Kajikawa2012}
Kajikawa, Y., and B.~Wang (2012), {Interdecadal Change of the South China Sea
  Summer Monsoon Onset}, \textit{J. Clim.}, \textit{25}, 3207--3218,
  \doi{10.1175/JCLI-D-11-00207.1}.

\bibitem[{\textit{Kang and Lu}(2012)}]{Kang2012}
Kang, S.~M., and J.~Lu (2012), {Expansion of the Hadley Cell under Global
  Warming: Winter versus Summer}, \textit{J. Clim.}, \textit{25}, 8387--8393,
  \doi{10.1175/JCLI-D-12-00323.1}.

\bibitem[{\textit{Kosaka et~al.}(2011)\textit{Kosaka, Xie, and
  Nakamura}}]{Kosaka2011}
Kosaka, Y., S.-P. Xie, and H.~Nakamura (2011), {Dynamics of Interannual
  Variability in Summer Precipitation over East Asia}, \textit{J. Clim.},
  \textit{24}, 5435--5453, \doi{10.1175/2011JCLI4099.1}.

\bibitem[{\textit{Kosaka et~al.}(2012)\textit{Kosaka, Chowdary, Xie, Min, and
  Lee}}]{Kosaka2012}
Kosaka, Y., J.~S. Chowdary, S.-P. Xie, Y.-M. Min, and J.-Y. Lee (2012),
  {Limitations of Seasonal Predictability for Summer Climate over East Asia and
  the Northwestern Pacific}, \textit{J. Clim.}, \textit{25}, 7574--7589,
  \doi{10.1175/JCLI-D-12-00009.1}.

\bibitem[{\textit{Kwon et~al.}(2007)\textit{Kwon, Jhun, and Ha}}]{Kwon2007}
Kwon, M., J.-G. Jhun, and K.-J. Ha (2007), {Decadal change in east Asian summer
  monsoon circulation in the mid-1990s}, \textit{Geophys. Res. Lett.},
  \textit{34}, L21,706, \doi{10.1029/2007GL031977}.

\bibitem[{\textit{Lei et~al.}(2014)\textit{Lei, Hoskins, and Slingo}}]{Lei2014}
Lei, Y., B.~Hoskins, and J.~Slingo (2014), {Natural variability of summer
  rainfall over China in HadCM3}, \textit{Clim. Dyn.}, \textit{42}, 417--432,
  \doi{10.1007/s00382-013-1726-8}.

\bibitem[{\textit{Li and Zhang}(2014)}]{Li2014}
Li, L., and Y.~Zhang (2014), {Effects of Different Configurations of the East
  Asian Subtropical and Polar Front Jets on Precipitation during the Mei-Yu
  Season}, \textit{J. Clim.}, \textit{27}, 6660--6672,
  \doi{10.1175/JCLI-D-14-00021.1}.

\bibitem[{\textit{Liang and Wang}(1998)}]{Liang1998}
Liang, X.-Z., and W.-C. Wang (1998), {Associations between China monsoon
  rainfall and tropospheric jets}, \textit{Q. J. R. Meteorol. Soc.},
  \textit{124}, 2597--2623.

\bibitem[{\textit{Liu et~al.}(2012)\textit{Liu, Wu, Hong, Dong, Duan, Bao, and
  Zhou}}]{Liu2012a}
Liu, Y., G.~Wu, J.~Hong, B.~Dong, A.~Duan, Q.~Bao, and L.~Zhou (2012),
  {Revisiting Asian monsoon formation and change associated with Tibetan
  Plateau forcing: II. Change}, \textit{Clim. Dyn.}, \textit{39}, 1183--1195,
  \doi{10.1007/s00382-012-1335-y}.

\bibitem[{\textit{Lu et~al.}(2007)\textit{Lu, Vecchi, and Reichler}}]{Lu2007}
Lu, J., G.~A. Vecchi, and T.~Reichler (2007), {Expansion of the Hadley cell
  under global warming}, \textit{Geophys. Res. Lett.}, \textit{34}, L06,805,
  \doi{10.1029/2006GL028443}.

\bibitem[{\textit{Molnar et~al.}(2010)\textit{Molnar, Boos, and
  Battisti}}]{Molnar2010}
Molnar, P., W.~R. Boos, and D.~S. Battisti (2010), {Orographic Controls on
  Climate and Paleoclimate of Asia: Thermal and Mechanical Roles for the
  Tibetan Plateau}, \textit{Annu. Rev. Earth Planet. Sci.}, \textit{38},
  77--102, \doi{10.1146/annurev-earth-040809-152456}.

\bibitem[{\textit{Nagashima et~al.}(2011)\textit{Nagashima, Tada, Tani, Sun,
  Isozaki, Toyoda, and Hasegawa}}]{Nagashima2011}
Nagashima, K., R.~Tada, A.~Tani, Y.~Sun, Y.~Isozaki, S.~Toyoda, and H.~Hasegawa
  (2011), {Millennial-scale oscillations of the westerly jet path during the
  last glacial period}, \textit{J. Asian Earth Sci.}, \textit{40}, 1214--1220,
  \doi{10.1016/j.jseaes.2010.08.010}.

\bibitem[{\textit{Nagashima et~al.}(2013)\textit{Nagashima, Tada, and
  Toyoda}}]{Nagashima2013}
Nagashima, K., R.~Tada, and S.~Toyoda (2013), {Westerly jet-East Asian summer
  monsoon connection during the Holocene}, \textit{Geochemistry, Geophys.
  Geosystems}, \textit{14}, 5041--5053, \doi{10.1002/2013GC004931}.

\bibitem[{\textit{Nigam and Lindzen}(1989)}]{Nigam1989}
Nigam, S., and R.~S. Lindzen (1989), {The sensitivity of stationary waves to
  variations in the basic state zonal flow}, \textit{J. Atmos. Sci.},
  \textit{46}, 1746--1768.

\bibitem[{\textit{Nigam et~al.}(2013)\textit{Nigam, Zhao, and
  Zhou}}]{Nigam2013}
Nigam, S., Y.~Zhao, and T.~Zhou (2013), {The south-flood north-drought pattern
  over Eastern China and the drying of the Gangetic Plain}, in \textit{Climate
  Change: Multidecadal and Beyond}, vol.~1, edited by M.~W. {Michael Ghil,
  Mojif Latif} and C.~Chan, World Press.

\bibitem[{\textit{Park and Schubert}(1997)}]{Park1997}
Park, C.-K., and S.~D. Schubert (1997), {On the nature of the 1994 East Asian
  summer drought}, \textit{J. Clim.}, \textit{10}, 1056--1070.

\bibitem[{\textit{Park and An}(2014{\natexlab{a}})}]{Park2014a}
Park, J.-H., and S.-I. An (2014{\natexlab{a}}), {The impact of tropical western
  Pacific convection on the North Pacific atmospheric circulation during the
  boreal winter}, \textit{Clim. Dyn.}, \textit{43}, 2227--2238,
  \doi{10.1007/s00382-013-2047-7}.

\bibitem[{\textit{Park and An}(2014{\natexlab{b}})}]{Park2014}
Park, J.-H., and S.-I. An (2014{\natexlab{b}}), {Southward displacement of the
  upper atmosphere zonal Jet in the eastern north Pacific due to global
  warming}, \textit{Geophys. Res. Lett.}, \textit{41}, 7861--7867,
  \doi{10.1002/2014GL062175}.

\bibitem[{\textit{Plumb and Hou}(1992)}]{Plumb1992}
Plumb, R.~A., and A.~Y. Hou (1992), {Response of a Zonally Symmetric Atmosphere
  to Subtropical Thermal Forcing}, \textit{J. Atmos. Sci.}, \textit{49},
  1790--1799.

\bibitem[{\textit{Priv\'{e} and Plumb}(2007)}]{Prive2007}
Priv\'{e}, N.~C., and R.~A. Plumb (2007), {Monsoon Dynamics with Interactive
  Forcing. Part I: Axisymmetric Studies}, \textit{J. Atmos. Sci.}, \textit{64},
  1417--1430, \doi{10.1175/JAS3916.1}.

\bibitem[{\textit{Qu and Huang}(2012)}]{Qu2012}
Qu, X., and G.~Huang (2012), {Impacts of tropical Indian Ocean SST on the
  meridional displacement of East Asian jet in boreal summer}, \textit{Int. J.
  Climatol.}, \textit{32}, 2073--2080, \doi{10.1002/joc.2378}.

\bibitem[{\textit{Sampe and Xie}(2010)}]{Sampe2010}
Sampe, T., and S.-P. Xie (2010), {Large-Scale Dynamics of the Meiyu-Baiu
  Rainband: Environmental Forcing by the Westerly Jet}, \textit{J. Clim.},
  \textit{23}, 113--134, \doi{10.1175/2009JCLI3128.1}.

\bibitem[{\textit{Schiemann et~al.}(2009)\textit{Schiemann, L\"{u}thi, and
  Sch\"{a}r}}]{Schiemann2009}
Schiemann, R., D.~L\"{u}thi, and C.~Sch\"{a}r (2009), {Seasonality and
  Interannual Variability of the Westerly Jet in the Tibetan Plateau Region},
  \textit{J. Clim.}, \textit{22}, 2940--2957, \doi{10.1175/2008JCLI2625.1}.

\bibitem[{\textit{Song et~al.}(2014)\textit{Song, Zhou, and Qian}}]{Song2014}
Song, F., T.~Zhou, and Y.~Qian (2014), {Responses of East Asian summer monsoon
  to natural and anthropogenic forcings in the 17 latest CMIP5 models},
  \textit{Geophys. Res. Lett.}, \textit{41}, 596--603,
  \doi{10.1002/2013GL058705.Received}.

\bibitem[{\textit{Tian and Yasunari}(1998)}]{Tian1998}
Tian, S.-F., and T.~Yasunari (1998), {Climatological Aspects and Mechanism of
  Spring Persistent Rains over Central China}, \textit{J. Meteorol. Soc.
  Japan}, \textit{76}, 57--71.

\bibitem[{\textit{Wang and LinHo}(2002)}]{Wang2002}
Wang, B., and LinHo (2002), {Rainy Season of the Asian-Pacific Summer Monsoon},
  \textit{J. Climate}, \textit{15}, 386--398.

\bibitem[{\textit{Wang et~al.}(2009)\textit{Wang, Huang, Wu, Yang, Fu, and
  Kikuchi}}]{Wang2009}
Wang, B., F.~Huang, Z.~Wu, J.~Yang, X.~Fu, and K.~Kikuchi (2009), {Multi-scale
  climate variability of the South China Sea monsoon: A review}, \textit{Dyn.
  Atmos. Ocean.}, \textit{47}, 15--37, \doi{10.1016/j.dynatmoce.2008.09.004}.

\bibitem[{\textit{Xin et~al.}(2006)\textit{Xin, Yu, Zhou, and Wang}}]{Xin2006}
Xin, X., R.~Yu, T.~Zhou, and B.~Wang (2006), {Drought in late spring of South
  China in recent decades}, \textit{J. Clim.}, \textit{19}, 3197--3206,
  \doi{10.1175/JCLI3794.1}.

\bibitem[{\textit{Xu et~al.}(2009)\textit{Xu, Zipser, and Liu}}]{Xu2009}
Xu, W., E.~J. Zipser, and C.~Liu (2009), {Rainfall Characteristics and
  Convective Properties of Mei-Yu Precipitation Systems over South China,
  Taiwan, and the South China Sea. Part I: TRMM Observations}, \textit{Mon.
  Weather Rev.}, \textit{137}, 4261--4275, \doi{10.1175/2009MWR2982.1}.

\bibitem[{\textit{Xuan et~al.}(2011)\textit{Xuan, Zhang, and Sun}}]{Xuan2011}
Xuan, S., Q.~Zhang, and S.~Sun (2011), {Anomalous midsummer rainfall in Yangtze
  River-Huaihe River valleys and its association with the East Asia westerly
  jet}, \textit{Adv. Atmos. Sci.}, \textit{28}, 387--397,
  \doi{10.1007/s00376-010-0111-3}.

\bibitem[{\textit{Yang et~al.}(2002)\textit{Yang, Lau, and Kim}}]{Yang2002}
Yang, S., K.~Lau, and K.~Kim (2002), {Variations of the East Asian Jet Stream
  and Asian-��Pacific-��American Winter Climate Anomalies.}, \textit{J. Clim.},
  \textit{15}, 306--325.

\bibitem[{\textit{Yatagai et~al.}(2012)\textit{Yatagai, Kamiguchi, Arakawa,
  Hamada, Yasutomi, and Kitoh}}]{Yatagai2012}
Yatagai, A., K.~Kamiguchi, O.~Arakawa, A.~Hamada, N.~Yasutomi, and A.~Kitoh
  (2012), {APHRODITE: Constructing a Long-Term Daily Gridded Precipitation
  Dataset for Asia Based on a Dense Network of Rain Gauges}, \textit{Bull. Am.
  Meteorol. Soc.}, \textit{93}, 1401--1415, \doi{10.1175/BAMS-D-11-00122.1}.

\bibitem[{\textit{Yeh et~al.}(1959)\textit{Yeh, Dao, and Li}}]{Yeh1959}
Yeh, T.-C., S.-Y. Dao, and M.-T. Li (1959), {The Abrupt Change of Circulation
  over the Northern Hemisphere during June and October}, in \textit{The
  Atmosphere and the Sea in Motion}, edited by B.~Bolin, pp. 249--267,
  Rockefeller Institute Press.

\bibitem[{\textit{Yin}(1949)}]{Yin1949}
Yin, M.~T. (1949), {Synoptic-aerologic study of the onset of the summer monsoon
  over India and Burma}, \textit{J. Meteorol.}, \textit{6}, 393--400.

\bibitem[{\textit{Yu and Zhou}(2007)}]{Yu2007}
Yu, R., and T.~Zhou (2007), {Seasonality and Three-Dimensional Structure of
  Interdecadal Change in the East Asian Monsoon}, \textit{J. Clim.},
  \textit{20}, 5344--5355, \doi{10.1175/2007JCLI1559.1}.

\bibitem[{\textit{Yu et~al.}(2004)\textit{Yu, Wang, and Zhou}}]{Yu2004}
Yu, R., B.~Wang, and T.~Zhou (2004), {Tropospheric cooling and summer monsoon
  weakening trend over East Asia}, \textit{Geophys. Res. Lett.}, \textit{31},
  L22,212, \doi{10.1029/2004GL021270}.

\bibitem[{\textit{Yu et~al.}(2010)\textit{Yu, Li, Yuan, and Chen}}]{Yu2010}
Yu, R., J.~Li, W.~Yuan, and H.~Chen (2010), {Changes in characteristics of
  late-summer precipitation over eastern China in the past 40 years revealed by
  hourly precipitation data}, \textit{J. Clim.}, \textit{23}, 3390--3396,
  \doi{10.1175/2010JCLI3454.1}.

\bibitem[{\textit{Zhang et~al.}(1999)\textit{Zhang, Sumi, and
  Kimoto}}]{Zhang1999}
Zhang, R., A.~Sumi, and M.~Kimoto (1999), {A Diagnostic Study of the Impact of
  El Ni\~{n}o on the Precipitation in China}, \textit{Adv. Atmos. Sci.},
  \textit{16}, 229--241.

\bibitem[{\textit{Zhou et~al.}(2009)\textit{Zhou, Gong, Li, and Li}}]{Zhou2009}
Zhou, T., D.~Gong, J.~Li, and B.~Li (2009), {Detecting and understanding the
  multi-decadal variability of the East Asian Summer Monsoon - Recent progress
  and state of affairs}, \textit{Meteorol. Zeitschrift}, \textit{18}, 455--467,
  \doi{10.1127/0941-2948/2009/0396}.

\end{thebibliography}


%% ------------------------------------------------------------------------ %%
%
%  END ARTICLE
%
%% ------------------------------------------------------------------------ %%
\end{article}
%
%
%% Enter Figures and Tables here:
%
% DO NOT USE \psfrag or \subfigure commands.
%
% Figure captions go below the figure.
% Table titles go above tables; all other caption information
%  should be placed in footnotes below the table.
%


%%% Hovm�ller diagram of Meiyu latitude occupancy, 1951-2007. Produced by MATLAB scripts meiyufig1.m and meiyustats_compact.m.
\begin{figure}
\begin{center}
\noindent\includegraphics[width=30pc]{Figures/meiyu_hovmoller}
\caption{Hovm\"oller climatology of East Asian rainfall, 1951-2007, with important time periods marked as follows: 1 - Spring Rains; 2 - Pre-Meiyu; 3 - Meiyu; 4 - Post-Meiyu; 5 - Fall Rains. a) Precipitation averaged over the longitudes 100-123\textdegree E; b) Probability of occurrence of a rainband for each day and latitude (both primary and secondary, in percentage), smoothed in time with a 9-day running box filter; c) Probability of primary rainband occurrence and mean intensity (9-day running mean); d) The conditional probability of a secondary rainband given the presence of a primary rainband, as well as the mean tilt and length of primary rainband events (9-day running mean).}
\label{hov}
\end{center}
\end{figure}

%Climatology of rainfall stages including rainfall, jet and most likely rainband configuration, and longitudinal averages.
\begin{figure}
\noindent\includegraphics[width=36pc]{Figures/climo}
\caption{Climatology of East Asian rainfall stages showing rainfall (shading), jet kernel density (contours of probability density in units of $10^{-4}$) and most common rainband position during that stage. Sidebars shows, for each time period, the longitude average over 105-123$^{\circ}$E of rainfall (thin blue line, units of mm day$^{-1}$), jet kernel density (red line, units of $10^{-4}$) and rainband position (dashed black line, absolute probability in \%, 1-degree latitude smoothing). From the Pre-Meiyu to Post-Meiyu, a peak in preferred jet latitude consistently occurs 5 degrees north of a corresponding maximum in rainband frequency.}
\label{climo}
\end{figure}

%%% Changes in Meiyu and rainfall behavior between 1951-1979 and 1980-2007
\begin{figure}[htbp]
\begin{center}
\includegraphics[width=36pc]{Figures/changes}
\caption{a) 15-day running mean of the change in rainfall between 1951-1979 and 1980-07, with 95\%/99\% confidence level marked by single/double cross-hatches; b) 15-day running mean of the change in rainband frequency between 1951-1979 and 1980-07, with two-degree smoothing in latitude and confidence levels marked as in a). The significance of rainfall changes is calculated by a permutation method. Time periods are marked as in Figure 1.}
\label{changes}
\end{center}
\end{figure}

%2D spatial distribution of change showing a) full year b) Pre-Meiyu and c) Post-Meiyu
\begin{figure}
\noindent\includegraphics[width=36pc]{Figures/changes_2d}
\caption{a) Whole year mean rainfall change, showing the South Flood-North Drought pattern; b) Rainfall changes during the Pre-Meiyu (days 121-160) with contours of jet density change overlain; c) Same as c, but for the Post-Meiyu (days 201-273). Statistical significance at 95\%/99\% level overlain with single/double hatches. Sidebars show, for each time period, the longitude average over 105-123$^{\circ}$E of changes in rainfall (thin blue line, units of mm day$^{-1}$), jet kernel density (red line, units of $10^{-4}$) and rainband position (dashed black line, absolute probability in \%, 1-degree latitude smoothing).}
\label{changes_2d}
\end{figure}

%%% Changes in jet mean between 1951-1979 and 1980-2007 + scatter plots of jet and rainband monthly anomalies.
\begin{figure}[htbp]
\begin{center}
\includegraphics[width=42pc]{Figures/jet}
\caption{a) 7-day running mean latitude of the westerly jet in the region 90-130$^\circ$E for the years 1958-1979 (blue, solid) and 1980-2001 (red, dashed). Bootstrapped 95\% confidence intervals are shaded. Time periods: 2 - Pre-Meiyu; 3 - Meiyu; 4 - Post-Meiyu; b) Plot of monthly anomalies in rainband frequency versus monthly anomalies in jet latitude during days 121-150 (May) for 1958-1979 (blue X) versus 1980-2001 (red circle); c) Same as b), but showing 30-day anomalies of rainband latitudes during the Post-Meiyu (201-230 and 231-260, each set of 30 days treated as a separate point). Histograms of anomalies are also shown on the side of each figure.}
\label{jet_seasonal}
\end{center}
\end{figure}

\end{document}